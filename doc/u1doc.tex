\documentclass[12pt]{article}

\setlength{\topmargin}{-3pc}
\setlength{\evensidemargin}{-.5pc}
\setlength{\oddsidemargin}{-.5pc}
\setlength{\textwidth}{39pc}
\setlength{\textheight}{55.5pc}
\baselineskip=\normalbaselineskip
\renewcommand{\baselinestretch}{1.2}
\setlength{\parskip}{0.25\baselineskip}

\usepackage{mathrsfs,amsbsy,amssymb,latexsym,amsfonts,amsmath,amsthm}
\usepackage[nosort]{cite}

\usepackage{bm}
\usepackage{graphicx} %new
\allowdisplaybreaks[1]

\makeatletter
\catcode`\@=11
\@addtoreset{equation}{section}
\renewcommand{\theequation}{\thesection.\arabic{equation}}
\def\@seccntformat#1{\csname the#1\endcsname.~~}
\makeatother

\begin{document}
\begin{titlepage}
  \renewcommand{\thefootnote}{\fnsymbol{footnote}}
  % \begin{flushright}
  %   preprint****
  % \end{flushright}
  \vspace*{1.0cm}

\begin{center}
  {\Large \bf
    FT-HMC code for 2D $U(1)$
  }
  \vspace{1.0cm}

  \centerline{
    {Nobuyuki Matsumoto}% %
    \footnote{E-mail address: 
      nobuyuki.matsumoto@riken.jp}
  }

  % \vskip 0.8cm
  % {\it RIKEN/BNL Research center, Brookhaven National Laboratory,
  %   Upton, NY 11973, USA}
  \vskip 1.2cm

  \end{center}

  %%%%%%%%%%%%%%%%%%%%%%%%%%%%%%%%%%%%%%% 
  \begin{abstract}
    %%%%%%%%%%%%%%%%%%%%%%%%%%%%%%%%%%%%%%%
    Brief description of the code.
    Many features mimic the qlat \cite{qlat}.
    %%%%%%%%%%%%%%%%%%%%%%%%%%%%%%%%%%%%%%% 
  \end{abstract}
  %%%%%%%%%%%%%%%%%%%%%%%%%%%%%%%%%%%%%%% 
\end{titlepage}

\pagestyle{empty}
\pagestyle{plain}

\tableofcontents
\setcounter{footnote}{0}

\section{Physical system}
\label{sec:sys}

\noindent \underline{Path integral}\newline
$U(1)$ lattice gauge theory on $T^2$:
\begin{align}
  Z_{\beta,\theta} \equiv \int(dU)\, e^{-S(U) + i \theta Q(U)},
\end{align}
where $U_{x,\mu}$ are $U(1)$-valued link variables
(see figure~\ref{fig:lat}):
\begin{align}
  U_{x,\mu} = e^{i\phi_{x,\mu}}
\end{align}
\begin{figure}[htb]
  \centering
  \includegraphics[width=70mm]{lattice.pdf}
  \caption{2D $U(1)$ lattice.}
  \label{fig:lat}
\end{figure}
and $(dU)$ the Haar measure:
\begin{align}
  (dU) = \prod_{x,\mu} \Big( \frac{d\phi_{x,\mu}}{2\pi} \Big).
\end{align}
We choose as $S(U)$ the Wilson action \cite{Wilson:1974sk}:
\begin{align}
  S(U) \equiv
  -\beta \sum_x \cos \kappa_x
\end{align}
and $Q(U)$ the integer-valued topological charge \cite{Luscher:1981zq,Phillips:1985sgx}:
\begin{align}
  Q(U) \equiv \frac{1}{2\pi} \sum_x \kappa_x,
\end{align}
where $\kappa_x$ is the plaquette angle:
\begin{align}
  \kappa_x
  &\equiv {1\over i}\ln\big(
  U_{x,0} U_{x+0,1} U_{x+1,0}^\dagger U_{x,1}^\dagger
  \big)\nonumber\\
  &= \phi_{x,0} + \phi_{x+0,1} -\phi_{x+1,0} - \phi_{x,1} \quad ({\rm mod} \, 2\pi),
\end{align}
where it is assumed that we take the principal branch of the logarithm:
\begin{align}
  \kappa_x \in [-\pi,\pi).
\end{align}

Classically,
the gauge potential $A_{x,\mu}$ in the continuum is defined as:
\begin{align}
  \phi_{x,\mu} = a A_{x,\mu},
\end{align}
where $a$ is the lattice spacing.
By scaling $\beta$ as:
\begin{align}
  \beta = \frac{1}{(ag)^2},
\end{align}
we have
\begin{align}
  S(U) \sim {a^2 \over 2g^2 }\sum_x F_{01}^2 \sim {1\over 4g^2} \int d^2x \,  F_{\mu\nu}^2
\end{align}
and
\begin{align}
  Q(U) \sim {1\over 2\pi} a^2 \sum_x F_{01} \sim {1\over 2\pi} \int d^2x \, F_{01},
\end{align}
where
\begin{align}
  F_{\mu\nu} \equiv \partial_\mu A_\nu - \partial_\nu A_\mu.
\end{align}

% \noindent \underline{Lattice topological charge}\newline
% Integer type 
% Non-integer type:
% \begin{align}
%   Q^{({\rm sin})} &\equiv -\frac{1}{2\pi} \sum_x \sin \kappa_x \nonumber\\
%   & =-\frac{1}{2\pi} \sum_x {\rm Im}\,
%                     \big(
%                     U_{x,0} U_{x+0,1} U_{x+1,0}^\dagger U_{x,1}^\dagger
%                     \big).
% \end{align}

\noindent \underline{Exact solutions}\newline
We compute the partition function:
\begin{align}
  Z_{\beta,\theta}
  = \int (dU)\, \prod_{x} e^{\beta\cos\kappa_x +\frac{i\theta}{2\pi} \kappa_x}.
\end{align}
We can use the Fourier expansion:
\begin{align}
  e^{\beta \cos \kappa + \frac{i\theta}{2\pi} \kappa}
  = \sum_{n\in \mathbb{Z}} \lambda_n(\beta,\theta) e^{in \kappa},
\end{align}
where
\begin{align}
  \lambda_n(\beta,\theta)
  &\equiv \int_{-\pi}^\pi {d\kappa \over 2\pi}\,
    e^{-i \nu \kappa} e^{\beta \cos \kappa} \Big|_{\nu=n-\theta/2\pi} \label{eq:integral} \\
  &\sim
    \frac{e^\beta}{\sqrt{2\pi \beta}} \sum_{k\geq 0}
    (\nu,k)\Big(\frac{-1}{2\beta}\Big)^k
    \Big|_{\nu=n-\theta/2\pi}
    \label{eq:asymptotic}
\end{align}
with
\begin{align}
  (\nu,k) \equiv \frac{\Gamma(\nu+k+1/2)}{k!\, \Gamma(\nu-k+1/2)}
  = \frac{\{4\nu^2-1^2\}\{4\nu^2-3^2\}\cdots\{4\nu^2-(2k-1)^2\}}{2^{2k}\cdot k!}.
\end{align}
For $\theta=0$,
$\lambda_n(\beta,\theta)$ is identical to the modified Bessel function $I_\nu(\beta)$,
but they are different generically.\footnote{
  The integral~\eqref{eq:integral} is an even function of $\nu$,
  but $I_\nu(\beta)$ is not.
}
We now have
\begin{align}
  Z_{\beta,\theta}
  = \int (dU)\, \prod_x \sum_{n_x}
    \lambda_{n_x}(\beta,\theta)
    e^{i n_x \kappa_x}
  = \sum_n \lambda_n(\beta, \theta)^{\hat{V}},
\end{align}
where $\hat{V}$ is the lattice volume (in the lattice units).
% It is a good approximation for large $\beta$:
% \begin{align}
%   Z_{\beta,\theta} \sim 
% \end{align}

We compute some expectation values for the case $\theta=0$.
The average plaquette can be calculated as
\begin{align}
  \langle \cos \kappa_x \rangle
  &=
    { 1\over \hat{V}} \frac{d}{d\beta} \ln Z_{\beta,\theta=0} \nonumber\\
  &=
    \frac{ \sum_n \big[(I_{n+1}(\beta) + I_{n-1}(\beta))/2\big]
    \cdot I_n(\beta)^{\hat{V}-1}}
    {\sum_n I_n(\beta)^{\hat{V}}} \nonumber\\
  &\to
    \frac{I_1(\beta)}{I_0(\beta)}
    \sim \Big(1-\frac{1}{2\beta}\Big) \quad (\beta\to \infty),
\end{align}
where we used that $(0,0)=(1,0)=1$, $(0,1)=-1/4$ and $(1,1)=3/4$.
We define the average energy density by:
\begin{align}
  e \equiv \frac{1}{a^4 \hat{V}} \sum_x (1-\cos \kappa_x)
  \to \frac{1}{2V} \int d^2x\, F_{01}^2 \quad (a\to 0),
\end{align}
where $V\equiv a^2 \hat{V}$ is the physical volume of the system.
This is actually a divergent quantity as can be seen from that:
\begin{align}
  \langle e \rangle \sim \frac{g^2}{2a^2}.
\end{align}
% The UV fluctuation is so strong that the energy is divergent.
% This is reflected to the fact that
It is also notable that
there is no correlation length in this system,
which can be seen from the above calculation by
modifying $\beta$ to $\beta_x$ and by taking
the derivatives with respect to $\beta_x$:
\begin{align}
  &\langle \cos^2 \kappa_x \rangle
  =
    \frac{ \sum_n \big[(I_{n+2}(\beta) + I_{n}(\beta)) + I_{n-2}(\beta))/4\big]
    \cdot I_n(\beta)^{\hat{V}-1}}
    {\sum_n I_n(\beta)^{\hat{V}}}, \\
  &\langle \cos \kappa_x \cos \kappa_y \rangle
  =
    \frac{ \sum_n \big[(I_{n+1}(\beta) + I_{n-1}(\beta))^2/4\big]
    \cdot I_n(\beta)^{\hat{V}-2}}
    {\sum_n I_n(\beta)^{\hat{V}}} \quad (x\neq y).
\end{align}

We also calculate the susceptibility:
\begin{align}
  \chi_Q
  & \equiv {1\over V} \langle Q^2 \rangle \nonumber\\
  &=
    -{ 1\over V} \frac{d^2}{d\theta^2} \ln Z_{\beta,\theta}\Big|_{\theta=0}
    \nonumber\\
  &= -{ 1\over V} \frac{1}{Z_{\beta,\theta}} \hat{V}
    \sum_n \lambda_n^{\hat{V}-1} \cdot \frac{d^2\lambda_n}{d\theta^2}\Big|_{\theta=0} \nonumber\\
  &\sim
    -{ 1\over a^2}
    {1\over\lambda_0(\beta,\theta)}
    \frac{d^2\lambda_0}{d\theta^2}
    \Big|_{\theta=0},
\end{align}
where we used the fact $(d\lambda_n/d\theta) |_{\theta=0} = 0$.
To get the continuum limit, we take
the derivative of eq.~\eqref{eq:asymptotic} with respect to $\nu$.
Using the polygamma function:
\begin{align}
  \psi^{(m)}(z) \equiv \frac{d^m}{dz^m} \ln \Gamma(z),
\end{align}
we have:
\begin{align}
  &(2\pi)\frac{d}{d\theta}\lambda_n(\beta,\theta)
    \sim
    \frac{e^\beta}{\sqrt{2\pi \beta}} \sum_{k\geq 0}
    (\nu,k) A(\nu,k)
    \Big(\frac{-1}{2\beta}\Big)^k
    \Big|_{\nu=n-\theta/2\pi}, \\
  &(2\pi)^2\frac{d^2}{d\theta^2}\lambda_n(\beta,\theta)
    \sim
    \frac{e^\beta}{\sqrt{2\pi \beta}} \sum_{k\geq 0}
    (\nu,k)
    \big\{
    A(\nu,k)^2 + B(\nu,k)
    \big\}
    \Big(\frac{-1}{2\beta}\Big)^k
    \Big|_{\nu=n-\theta/2\pi}
\end{align}
with
\begin{align}
  A(\nu, k) &\equiv \psi^{(0)}(\nu+k+1/2)-\psi^{(0)}(\nu+k-1/2), \\
  B(\nu, k) &\equiv \psi^{(1)}(\nu+k+1/2)-\psi^{(1)}(\nu+k-1/2).
\end{align}
Therefore,
\begin{align}
  \chi_Q
  &\sim
  -{ 1\over a^2}
  {1\over I_0(\beta)}
  \frac{d^2\lambda_0}{d\theta^2}
    \Big|_{\theta=0} \nonumber\\
  &\sim
    -{ 1\over a^2} \cdot 1 \cdot
    \Big({-1\over 4}\Big)\cdot (-8) \cdot \Big({-1\over 2\beta}\Big)
    =g^2,
\end{align}
where we used that $A(0,0)=A(0,1)=0$, $B(0,0)=0$ and $B(0,1)=-8$.
The susceptibility has a finite value in the continuum limit.



\section{Field-transformed HMC}
\label{sec:ft-hmc}

The contents expressed here was
first developed by \cite{Luscher:2009eq} for $SU(3)$ on $T^4$.

\noindent \underline{Transformation}
\begin{align}
  U_{x,\mu} \to U_{x,\mu}^{(\epsilon)} \equiv \mathcal{F}_\epsilon(U)_{x,\mu}
  \equiv
  e^{\epsilon \tilde\partial_{x,\mu} K(U)} U_{x,\mu}.
  \label{eq:trsf}
\end{align}
We choose the kernel $K(U)$ to be the plaquettes:
\begin{align}
  K(U) = \sum_{x} \cos\kappa_{x}.
  \label{eq:tilde_s}
\end{align}
We recursively define for $\ell \geq 1$:
\begin{align}
  S_{\ell\epsilon}(U) \equiv
  S_{(\ell-1)\epsilon} (\mathcal{F}_\epsilon(U))
  - \ln \det \mathcal{F}_{\epsilon*}(U),
  \label{eq:action_ell}
\end{align}
where $S_0(U)=S(U)$
and $\mathcal{F}_{\epsilon *}(U)$ is
the Jacobian of the transformation~\eqref{eq:trsf}.
By acting
$d\equiv \sum_{x,\mu}d\phi_{x,\mu} \partial_{x,\mu}$
to eq.~\eqref{eq:trsf}, we find
\begin{align}
  \mathcal{F}_{\epsilon *}(U)_{x,\mu|y,\nu}
  =\delta_{xy}\delta_{\mu\nu}
  + \epsilon \partial_{y,\nu} \partial_{x,\mu} K(U).
  \label{eq:jacobian}
\end{align}
To the leading order in $\epsilon$,
\begin{align}
  S_{\ell\epsilon}(U) =
  S_{(\ell-1)\epsilon}(U)
  +
  \epsilon \sum_{x,\mu}
  \partial_{x,\mu}K(U) \partial_{x,\mu}S(U)
  -
  \epsilon \sum_{x,\mu}
  \partial_{x,\mu}^2 K(U).
\end{align}
Explicitly,
\begin{align}
  S_{\ell\epsilon}(U)
  =
  S_{(\ell-1)\epsilon}(U) + \epsilon \sum_{x} \Big[
  4 \beta \sin^2\kappa_x
  - \beta \sum_{\mu=\pm 0,\pm1} \sin\kappa_{x+\mu} \sin \kappa_x
  + 4 \cos\kappa_x
  \Big].
\end{align}
Note that, from eq.~\eqref{eq:jacobian},
\begin{align}
  \partial_{y,\nu}
  =\partial_{x,\mu}^{(\epsilon)}
  \mathcal{F}_{\epsilon *}(U)_{x,\mu|y,\nu}.
  \label{eq:deriv_law}
\end{align}
Therefore, from eq.~\eqref{eq:action_ell},
\begin{align}
  \partial_{y,\nu} S_{\ell \epsilon}(U)
  &=
    \sum_{x,\mu}
    \big[\delta_{xy}\delta_{\mu\nu}
    + \epsilon \partial_{y,\nu} \partial_{x,\mu}
    K(U)
    \big]
    \partial_{x,\mu}^{(\epsilon)}
    S_{(\ell-1) \epsilon}(U^{(\epsilon)}) 
    - {\rm tr} \big[\mathcal{F}_{\epsilon *}(U)^{-1}
    \partial_{y,\nu} \mathcal{F}_{\epsilon *}(U) \big].
    \label{eq:force_prop2}
\end{align}

In practice, we deal with finite $\epsilon$.
To ensure that the map is invertible,
we perform the even/odd partitioning.
In terms of the angle variables,
the transformation can be rewritten as:
\begin{align}
  \phi_{x,\mu}^{(\epsilon)} =
  \phi_{x,\mu}
  -
  \epsilon (\sin \kappa_{x'} - \sin \kappa_{x''}), \label{eq:thetaeps}
\end{align}
where
\begin{align}
  x' &= x,  \, x'' = x-1 \quad (\mu=0) \\
  x' &= x-0, \, x'' = x \quad (\mu=1).
\end{align}
To invert eq.~\eqref{eq:thetaeps},
we rewrite the equation as:
\begin{align}
  f(X) = Z_{x,\mu}(\phi)
  |_{\phi = \phi^{(\epsilon)}-\epsilon X}, \quad
  X=f(X),
\end{align}
where
\begin{align}
  Z_{x,\mu}(\phi) \equiv
  \partial_{x,\mu} K(U)
  = -\sin \kappa_{x'} + \sin \kappa_{x''}.
\end{align}
Then, for $\epsilon<1/2$ the map $f$ defines
the contraction mapping because:
\begin{align}
  |f(X)-f(Y)|
  &=
    |-\sin(\varphi+\epsilon X)+\sin(\varphi'-\epsilon X)
    +\sin(\varphi+\epsilon Y)-\sin(\varphi'-\epsilon Y)| \nonumber\\
  &=
    \Big|-2 \cos\Big(\varphi + \frac{\epsilon}{2}(X-Y)\Big) \sin\Big(\frac{\epsilon}{2}(X-Y)\Big) \\
  &~~~~~~~~~~~~~~
    + 2 \cos\Big(\varphi' + \frac{\epsilon}{2}(X-Y)\Big) \sin\Big(\frac{\epsilon}{2}(X-Y)\Big)\Big| \nonumber\\
  \leq 2 \epsilon |X-Y|.
\end{align}
% Under the partitioning,
% the force coming from the Jacobian can be simplified to:
% \begin{align}
%   {\rm tr} \big[\mathcal{F}_{\epsilon *}(U)^{-1} \partial_{y,\nu} \mathcal{F}_{\epsilon *}(U) \big]
%   &=
%     [1 + \epsilon (\cos \kappa_{y'}+\cos \kappa_{y''})]^{-1}
%     \epsilon (\sin \kappa_{y'}-\sin \kappa_{y''})
%     \nonumber\\
%   &+
%     \sum_{\kappa_{y'}\ni(x,\mu)\neq(y,\nu)}[1 + \epsilon (\cos \kappa_{x'}+\cos \kappa_{x''})]^{-1}
%     (-1)^\Delta\epsilon \sin \kappa_{y'}
%     \nonumber\\
%   &+
%     \sum_{\kappa_{y''}\ni(x,\mu)\neq(y,\nu)}[1 + \epsilon (\cos \kappa_{x'}+\cos \kappa_{x''})]^{-1}
%     (-1)^\Delta\epsilon \sin \kappa_{y''}.
% \end{align}


\section{Code}
\label{sec:code}

The code consists of the {\tt main.c} and the following
header files:
\begin{description}
\item[util.h] some utility functions (print functions for vectors, projection from $\mathbb{R}$ to $U(1)$).
\item[lattice.h] defines the {\tt Lattice} class, which describes the geometry of the lattice.
\item[coord.h] defines the {\tt Coord} class, which describes a point $x$ on the lattice.
\item[scalar\_field.h] defines the {\tt ScalarField} class, which describes the field $\phi_{x,\mu}$ on the lattice. We use it for the angular variables.
\item[action.h] defines the {\tt WilsonAction} class, which describes the action function $S$.
\item[corr.h] defines the {\tt Corr} class, which gives the correlator functions $\cos\kappa_x\cdot\cos\kappa_y$.
\item[rnd.h] defines the {\tt Rnd} class, which gives a set of mt19937 generators on each site or link.
\item[hmc.h] defines the {\tt HMC} class, which gives the {\it ordinary} HMC algorithm.
\item[kernel.h] defines the {\tt Kernel} class, which describes the kernel $K$ for the field transformations.
\item[field\_trsf.h] defines the {\tt FieldTrsf} class, which describes the infinitesimal map $\cal{F}_\epsilon$.
\item[ft\_hmc.h] defines the {\tt FT\_HMC} class, which gives the FT-HMC algorithm.
\end{description}

The present code calculates and records the
\begin{itemize}
\item average plaquette
\item two-point functions
\item topological charge.
\end{itemize}
The plaquette and the topological charge are recorded for both the physical field $U$ and the auxiliary field $V$.

The code is aimed to be run on laptops and uses {\tt c++17} for {\tt filesystem}.
The remaining part only uses {\tt c++11}.
It can run with openmp. No MPI calls.
It generates {\tt 500 configurations} with two-step Wilson-flowed HMC in {\tt 30 sec} for $8\times 8$ lattice on my workstation (with OpenMP).
It creates {\tt ./result/} directory, in which the observable data are stored, and a few log files.
The contents written in this paragraph are mostly described in {\tt main.c}.

One can compile and run the code with {\tt make IS\_FTHMC=0; make run}.
The included {\tt analysis.ipynb} can be used to analyze observable data. The exact solutions (or their approximated functions) are written here for the checks.



% \section*{Acknowledgments}
% \appendix




%%%%%%%%%%%%%%%%%%%%%%%%%%%%%%%%%%%%%%% 
\baselineskip=0.9\normalbaselineskip
%%%%%%%%%%%%%%%%%%%%%%%%%%%%%%%%%%%%%%% 

%%%%%%%%%%%%%%%%%%%%%%%%%%%%%%%%%%%%%%% 
%%%%%%%%%%%%%%%%%%%%%%%%%%%%%%%%%%%%%%% 
\begin{thebibliography}{99}
  \setlength{\itemsep}{-2pt}
  %%%%%%%%%%%%%%%%%%%%%%%%%%%%%%%%%%%%%%% 
  %%%%%%%%%%%%%%%%%%%%%%%%%%%%%%%%%%%%%%% 

  % \cite{qlat}
\bibitem{qlat}
  L.~Jin,
  ``Qlattice,''
  https://github.com/jinluchang/Qlattice

%\cite{Wilson:1974sk}
\bibitem{Wilson:1974sk}
K.~G.~Wilson,
``Confinement of Quarks,''
Phys. Rev. D \textbf{10}, 2445-2459 (1974)
doi:10.1103/PhysRevD.10.2445
%5938 citations counted in INSPIRE as of 16 Jan 2023

%\cite{Luscher:1981zq}
\bibitem{Luscher:1981zq}
M.~Luscher,
``Topology of Lattice Gauge Fields,''
Commun. Math. Phys. \textbf{85}, 39 (1982)
doi:10.1007/BF02029132
%299 citations counted in INSPIRE as of 10 Jan 2023

%\cite{Phillips:1985sgx}
\bibitem{Phillips:1985sgx}
A.~Phillips,
``Characteristic Numbers of U(1) Valued Lattice Gauge Fields,''
Annals Phys. \textbf{161}, 399-422 (1985)
doi:10.1016/0003-4916(85)90086-7
%27 citations counted in INSPIRE as of 10 Jan 2023

%\cite{Luscher:2009eq}
\bibitem{Luscher:2009eq}
M.~Luscher,
``Trivializing maps, the Wilson flow and the HMC algorithm,''
Commun. Math. Phys. \textbf{293}, 899-919 (2010)
doi:10.1007/s00220-009-0953-7
[arXiv:0907.5491 [hep-lat]].
%291 citations counted in INSPIRE as of 17 Jan 2023


\end{thebibliography}
%%%%%%%%%%%%%%%%%%%%%%%%%%%%%%%%%%%%%%% 
%%%%%%%%%%%%%%%%%%%%%%%%%%%%%%%%%%%%%%% 

\end{document}
